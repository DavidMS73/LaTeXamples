%----------------------------------------------------
% Desigualdades e inecuaciones
% MSc. Fausto M. Lagos Suárez
% @piratax007 - V1.1.3 - 2019
% LPPL v1.3c
%----------------------------------------------------

% Paquetes
\usepackage{amsgen,amsmath,amstext,amsbsy,amsopn,tikz,amssymb,tkz-linknodes}
\usepackage{xstring}

% Colores
\definecolor{myBlue}{HTML}{027FDF}
\definecolor{negative}{HTML}{181818}
\definecolor{positive}{HTML}{AA3939}

% Recta Real
\newcommand{\reaLine}[2]{
  \node (li) at (#1 - 1.5, 0) {};
  \node (ls) at (#2 + 1.5, 0) {$\mathbb{R}$};
	\draw [>=stealth, <->] (li) -- (ls);
}

%----------------------------------------------------
% Intervalo
% #1 cota inferior
% #2 cota superior
% #3 posición vertical del intervalo
% #4 fill o fill = none o <- o ->, tipo de cota
% #5 fill o fill = none, tipo de cota
% #6 nonInf o inf, tipo de intervalo
% #7 color
\newcommand{\interval}[7]{
    \IfEqCase {#6}{
        {nonInf}{
          \node [circle, draw, #4, line width = 1.5pt, color = #7, inner sep = 0pt, minimum size = 5pt] (ci) at (#1, #3) {};
          \node [circle, draw, #5, line width = 1.5pt, color = #7, inner sep = 0pt, minimum size = 5pt] (cs) at (#2, #3) {};
          \draw [line width = 1.5pt, color = #7] (ci) -- (cs);
          \draw (ci) -- (#1, -.2);
          \node (tag) at (#1, -.4) {#1};
          \draw (cs) -- (#2, -.2);
          \node (tag) at (#2, -.4) {#2};
        }{inf}{
        	  \IfEqCase {#4}{
            	{<-}{
			        \node [circle, draw, #5, line width = 1.5pt, color = #7, inner sep = 0pt, minimum size = 5pt] (cs) at (#2, #3) {};
                    \draw [>=stealth, #4, line width = 1.5pt, color = #7] (#1 - 1.5, #3) -- (cs);
                    \draw (cs) -- (#2, -.2);
                    \node (tag) at (#2, -.4) {#2};
                }{->}{
                	\node [circle, draw, #5, line width = 1.5pt, color = #7, inner sep = 0pt, minimum size = 5pt] (ci) at (#1, #3) {};
                    \draw [>=stealth, ->, line width = 1.5pt, color = #7] (ci) -- (#2 + 1.5, #3);
                    \draw (ci) -- (#1, -.2);
                    \node (tag) at (#1, -.4) {#1};
                }
            }
        }
        }[\PackageError{tree}{Undefined option to intervals: #6}{}]
}
%----------------------------------------------------

%----------------------------------------------------
% Cementerio

% Valores de referencia
% #1 lista de los valores de referencia - entrecomillados y separados por coma
% #2 número de valores de referencia en #1, contados desde cero
% #3 número de inecuaciones
\newcommand{\refVals}[3]{
  \def\rV{{#1}}
  \foreach \i in {0,1,...,#2}{
    \pgfmathsetmacro{\rVIndex}{\rV[\i]}
    \coordinate (c\i) at (\rVIndex, 0);
    \draw[gray!75] (\rVIndex, -.2) node[below] {\small{\rVIndex}} -- (\rVIndex, #3*.75);
  }
}

% Inecuación
% #1 lista de signos de la inecuación - entrecomillados y separados por coma
% #2 número de signos en #1, contados desde cero
% #3 número de la inecuación numerada de abajo para arriba
% #4 la inecuación en escritura matemática
\newcommand{\inequality}[4]{
  \def\sI{{#1}}
  \node[left] at ($(li) + (0, #3*.5)$) {\tiny{#4}};
  \draw[>=stealth, <->, gray!75] ($(li) + (0, #3*.5)$) -- ($(ls) + (0, #3*.5)$);
  \foreach \i in {0,1,...,#2}{
    \pgfmathsetmacro{\sIndex}{\sI[\i]}
    \IfEq {\i}{0}{
        \path ($(li)+(0, #3*.5)$) -- ($(c0)+(0, #3*.5)$) node[midway, above, myBlue] {$\sIndex$};
      }{\IfEq {\i}{#2}{
        \path ($(c\the\numexpr\i-1)+(0, #3*.5)$) -- ($(ls)+(0, #3*.5)$) node[midway, above, myBlue] {$\sIndex$};
      }{\path ($(c\the\numexpr\i-1)+(0, #3*.5)$) -- ($(c\i)+(0, #3*.5)$) node[midway, above, myBlue] {$\sIndex$};}
      }
    }
}

% Respuesta
\newcommand{\solution}[3]{
  \def\sI{{#1}}
  \node[left] at (li) {\tiny{#3}};
  \foreach \i in {0,1,...,#2}{
    \pgfmathsetmacro{\sIndex}{\sI[\i]}
    \IfEq {\i}{0}{
        \path (li) -- (c0) node[midway, above, magenta!75] {$\sIndex$};
      }{\IfEq {\i}{#2}{
        \path (c\the\numexpr\i-1) -- (ls) node[midway, above, magenta!75] {$\sIndex$};
      }{\path (c\the\numexpr\i-1) -- (c\i) node[midway, above, magenta!75] {$\sIndex$};}
      }
    }
}
%----------------------------------------------------