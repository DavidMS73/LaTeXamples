\documentclass{article}

\RequirePackage[utf8]{inputenc}
\RequirePackage[spanish]{babel}
\RequirePackage{amsmath, amsfonts, amssymb}
\RequirePackage{minted}
\RequirePackage{subcaption} 
\RequirePackage{wrapfig}
\RequirePackage{mdframed}
\RequirePackage{listings}
\RequirePackage{etex}
\RequirePackage{fancyhdr}
\RequirePackage{graphics}
\RequirePackage{xcolor}
\RequirePackage{tikz}
\usetikzlibrary{shadows}
\lfoot{Este ejemplo esta puesto a disposición mediante la licencia GPLv3, usted es libre de usarlo, estudiarlo, modificarlo o compartirlo siempre que las obras derivadas las comparta bajo los mismos términos y mencione al autor original.}
\pagenumbering{gobble}
\renewcommand{\headrulewidth}{0pt}

\graphicspath{{./images/}}

%%------------------------------------------
%% Márgenes
%%------------------------------------------
\RequirePackage[top = 1.5cm, bottom = 1.5cm, left = 1.5cm, right = 1.5cm, foot = .5cm]{geometry}
%%------------------------------------------

%%----------------------------------------------------------------------
%% Configuración global del ambiente minted para LaTeX
%%----------------------------------------------------------------------
\setminted[latex]{
style = friendly, 
frame = lines, 
bgcolor = codeback!10, 
linenos = true,
numberblanklines = false,
tabsize = 2, 
fontsize = \small,
resetmargins, 
framesep = 2mm, 
baselinestretch = 1.2,
numbersep = 3pt
}
%%----------------------------------------------------------------------

%%----------------------------------------------------------------------
%% Entradas:
%% #1 un color
%% #2 Un texto
%%----------------------------------------------------------------------
%% Salida: El texto #2 escrito en negrita y color #1
%%----------------------------------------------------------------------
\newcommand{\bfcolor}[2]{
\textbf{\textcolor{#1}{#2}}
}
%%----------------------------------------------------------------------
  
%%----------------------------------------------------------------------
%% Entradas:
%% #1 Un comando
%% Salida:
%%----------------------------------------------------------------------
%% El comando #1 en formato verbatim y negrilla
%%----------------------------------------------------------------------
\newcommand{\bftt}[1]{\textbf{\texttt{#1}}}
%%----------------------------------------------------------------------

%%----------------------------------------------------------------------
%% Entrada:
%% #1 Un caracter del teclado
%%----------------------------------------------------------------------
%% Salida:
%% La representación de la tecla
%%----------------------------------------------------------------------
\newcommand*\keystroke[1]{
  \tikz[baseline=(key.base)]
    \node[%
      draw,
      fill=white,
      drop shadow={shadow xshift=0.25ex,shadow yshift=-0.25ex,fill=black,opacity=0.75},
      rectangle,
      rounded corners=2pt,
      inner sep=1pt,
      line width=0.5pt,
      font=\scriptsize\sffamily
    ](key) {#1\strut}
  ;
}
%%----------------------------------------------------------------------

%%----------------------------------------------------------------------
%% Entradas:
%% #1 un caracter del teclado
%% Salida:
%% La representación de la tecla en negrilla
%%----------------------------------------------------------------------
\newcommand{\keystrokebftt}[1]{\keystroke{\bftt{#1}}}
%%----------------------------------------------------------------------

%%----------------------------------------------------------------------
%% Escribir un comando (LaTeX) no precedido por backslash
%%----------------------------------------------------------------------
\newcommand{\cmd}[1]{{\color[HTML]{008000}\bftt{#1}}} 
%%----------------------------------------------------------------------

%%----------------------------------------------------------------------
%% Escribe el caracter backslash
%%----------------------------------------------------------------------
\newcommand{\bs}{\char`\\}
%%----------------------------------------------------------------------

%%----------------------------------------------------------------------
%% Escribe un comando (LaTeX) precedido de backslash
%%----------------------------------------------------------------------
\newcommand{\cmdbs}[1]{\cmd{\bs{}#1}} 
%%----------------------------------------------------------------------

%%----------------------------------------------------------------------
%% Comandos requeridos por \cmdbegin y \cmdend
%%----------------------------------------------------------------------
\newcommand{\lcb}{\char '173}
\newcommand{\rcb}{\char '175}
%%----------------------------------------------------------------------

%%----------------------------------------------------------------------
%% Escriben el inicio y fin - respectivamente - de un ambiente LaTeX
%%----------------------------------------------------------------------
\newcommand{\cmdbegin}[1]{\cmdbs{begin\lcb}\bftt{#1}\cmd{\rcb}}
\newcommand{\cmdend}[1]{\cmdbs{end\lcb}\bftt{#1}\cmd{\rcb}}
%%----------------------------------------------------------------------

%%----------------------------------------------------------------------
%% Ambiente para generar código LaTeX y su resultado uno al lado del 
%% otro enmarcados en un frame sensillo
%%----------------------------------------------------------------------
\newenvironment{latexPlusCodeLeftFrame}
  {\VerbatimEnvironment
   \begin{VerbatimOut}{example.out}}
  {\end{VerbatimOut}
   \setlength{\parindent}{0pt}
   \fbox{\begin{tabular}{l|l}
   \begin{minipage}{0.55\linewidth}
     \inputminted[fontsize=\scriptsize,resetmargins]{latex}{example.out}
   \end{minipage} &
   \begin{minipage}{0.35\linewidth}
     \setlength{\parskip}{6pt plus 1pt minus 1pt}%
     \raggedright\scriptsize\input{example.out}
   \end{minipage}
   \end{tabular}}}
%%----------------------------------------------------------------------

%%----------------------------------------------------------------------
%% Idéntico al anterior pero sin frame
%%----------------------------------------------------------------------
\newenvironment{latexPlusCodeLeftnoFrame}
  {\VerbatimEnvironment
   \begin{VerbatimOut}{example.out}}
  {\end{VerbatimOut}
   \setlength{\parindent}{0pt}
   \begin{tabular}{ll}
   \begin{minipage}{0.55\linewidth}
     \inputminted[fontsize=\scriptsize,resetmargins]{latex}{example.out}
   \end{minipage} &
   \begin{minipage}{0.35\linewidth}
     \setlength{\parskip}{6pt plus 1pt minus 1pt}%
     \raggedright\scriptsize\input{example.out}
   \end{minipage}
   \end{tabular}}
%%----------------------------------------------------------------------

%%----------------------------------------------------------------------
%% Ambiente para generar el resultado de un código LaTeX seguido por 
%% el código
%%----------------------------------------------------------------------
\newenvironment{latexPlusCodeBelow}
  {\VerbatimEnvironment
   \begin{VerbatimOut}{example.out}}
  {\end{VerbatimOut}
		\input{example.out}
    \inputminted[fontsize=\scriptsize,resetmargins]{latex}{example.out}
   }
%%----------------------------------------------------------------------

%%------------------------------------------
%% Colores
%%------------------------------------------
\definecolorset{HTML}{}{}{ %
	color1,423E3A;color2,ACB8F1;color3,FF1E12;color4,FF7600;color5,FF8012;myBlue,027FDF; %
	myBlack,181818;myRed,AA3939;myGreen,19B170;myGray,181818;myOrange,FF6302;negative,181818; %
	positive,AA3939;enlace,1D8AEC;url,000000;youtube,C4302B;codeback,A4DCF2;codeframe,0B2C39; %
	structColor,201090;barColor,202122;backColor,8998B5;titlesColor,008100;blueR,1042A6; %
	blueOctave,015794;hlOrange,FFAD7E;citeColor,AA3939 %
}

\definecolorset{rgb}{}{}{ %
	signBlue,0, .25, .53;emeraldGreen,0, .79, .34;topaz,0, .6, .88; %
	indigoDye,.05, .31, .55;indigo2,.13, .53, .41;blueSpider,.15, .27, .43; %
	darkOliveGreen2,.74, .93, .41;ochre,.8, .47, .13;darkOrange,.8, .4, .0; %
	skyblue6,.2, .6, .8;firebrick,.7, .13, .13;blueice,.85, .96, .94; %
	lightcopper,.93, .76, .58
}
%%------------------------------------------

\author{MSc. Fausto M. Lagos S.}
\title{Manual de los macros \bftt{LaTeX2Code.tex}}
\date{\today}

\begin{document}
\maketitle

\begin{abstract}
	El conjunto de macros \bftt{LaTeX2Code.tex} permite documentar ejemplos de \LaTeX{} incluyendo el resultado del código y el código mismo en el documento pdf resultante. 
\end{abstract}
	
	\section{A Modo de Instalación}
		Para hacer uso de los macros de \bftt{LaTeX2Code} no hace falta desarrollar ningún proceso de instalación, simplemente lleve el archivo \bftt{LaTeX2Code.tex} a la raíz de su proyecto \LaTeX{} y agregue en el preámbulo \mintinline{latex}{\RequirePackage[utf8]{inputenc}
\RequirePackage[spanish]{babel}
\RequirePackage{amsmath, amsfonts, amssymb}
\RequirePackage{minted}
\RequirePackage{subcaption} 
\RequirePackage{wrapfig}
\RequirePackage{mdframed}
\RequirePackage{listings}
\RequirePackage{etex}
\RequirePackage{fancyhdr}
\RequirePackage{graphics}
\RequirePackage{xcolor}
\RequirePackage{tikz}
\usetikzlibrary{shadows}
\lfoot{Este ejemplo esta puesto a disposición mediante la licencia GPLv3, usted es libre de usarlo, estudiarlo, modificarlo o compartirlo siempre que las obras derivadas las comparta bajo los mismos términos y mencione al autor original.}
\pagenumbering{gobble}
\renewcommand{\headrulewidth}{0pt}

\graphicspath{{./images/}}

%%------------------------------------------
%% Márgenes
%%------------------------------------------
\RequirePackage[top = 1.5cm, bottom = 1.5cm, left = 1.5cm, right = 1.5cm, foot = .5cm]{geometry}
%%------------------------------------------

%%----------------------------------------------------------------------
%% Configuración global del ambiente minted para LaTeX
%%----------------------------------------------------------------------
\setminted[latex]{
style = friendly, 
frame = lines, 
bgcolor = codeback!10, 
linenos = true,
numberblanklines = false,
tabsize = 2, 
fontsize = \small,
resetmargins, 
framesep = 2mm, 
baselinestretch = 1.2,
numbersep = 3pt
}
%%----------------------------------------------------------------------

%%----------------------------------------------------------------------
%% Entradas:
%% #1 un color
%% #2 Un texto
%%----------------------------------------------------------------------
%% Salida: El texto #2 escrito en negrita y color #1
%%----------------------------------------------------------------------
\newcommand{\bfcolor}[2]{
\textbf{\textcolor{#1}{#2}}
}
%%----------------------------------------------------------------------
  
%%----------------------------------------------------------------------
%% Entradas:
%% #1 Un comando
%% Salida:
%%----------------------------------------------------------------------
%% El comando #1 en formato verbatim y negrilla
%%----------------------------------------------------------------------
\newcommand{\bftt}[1]{\textbf{\texttt{#1}}}
%%----------------------------------------------------------------------

%%----------------------------------------------------------------------
%% Entrada:
%% #1 Un caracter del teclado
%%----------------------------------------------------------------------
%% Salida:
%% La representación de la tecla
%%----------------------------------------------------------------------
\newcommand*\keystroke[1]{
  \tikz[baseline=(key.base)]
    \node[%
      draw,
      fill=white,
      drop shadow={shadow xshift=0.25ex,shadow yshift=-0.25ex,fill=black,opacity=0.75},
      rectangle,
      rounded corners=2pt,
      inner sep=1pt,
      line width=0.5pt,
      font=\scriptsize\sffamily
    ](key) {#1\strut}
  ;
}
%%----------------------------------------------------------------------

%%----------------------------------------------------------------------
%% Entradas:
%% #1 un caracter del teclado
%% Salida:
%% La representación de la tecla en negrilla
%%----------------------------------------------------------------------
\newcommand{\keystrokebftt}[1]{\keystroke{\bftt{#1}}}
%%----------------------------------------------------------------------

%%----------------------------------------------------------------------
%% Escribir un comando (LaTeX) no precedido por backslash
%%----------------------------------------------------------------------
\newcommand{\cmd}[1]{{\color[HTML]{008000}\bftt{#1}}} 
%%----------------------------------------------------------------------

%%----------------------------------------------------------------------
%% Escribe el caracter backslash
%%----------------------------------------------------------------------
\newcommand{\bs}{\char`\\}
%%----------------------------------------------------------------------

%%----------------------------------------------------------------------
%% Escribe un comando (LaTeX) precedido de backslash
%%----------------------------------------------------------------------
\newcommand{\cmdbs}[1]{\cmd{\bs{}#1}} 
%%----------------------------------------------------------------------

%%----------------------------------------------------------------------
%% Comandos requeridos por \cmdbegin y \cmdend
%%----------------------------------------------------------------------
\newcommand{\lcb}{\char '173}
\newcommand{\rcb}{\char '175}
%%----------------------------------------------------------------------

%%----------------------------------------------------------------------
%% Escriben el inicio y fin - respectivamente - de un ambiente LaTeX
%%----------------------------------------------------------------------
\newcommand{\cmdbegin}[1]{\cmdbs{begin\lcb}\bftt{#1}\cmd{\rcb}}
\newcommand{\cmdend}[1]{\cmdbs{end\lcb}\bftt{#1}\cmd{\rcb}}
%%----------------------------------------------------------------------

%%----------------------------------------------------------------------
%% Ambiente para generar código LaTeX y su resultado uno al lado del 
%% otro enmarcados en un frame sensillo
%%----------------------------------------------------------------------
\newenvironment{latexPlusCodeLeftFrame}
  {\VerbatimEnvironment
   \begin{VerbatimOut}{example.out}}
  {\end{VerbatimOut}
   \setlength{\parindent}{0pt}
   \fbox{\begin{tabular}{l|l}
   \begin{minipage}{0.55\linewidth}
     \inputminted[fontsize=\scriptsize,resetmargins]{latex}{example.out}
   \end{minipage} &
   \begin{minipage}{0.35\linewidth}
     \setlength{\parskip}{6pt plus 1pt minus 1pt}%
     \raggedright\scriptsize\input{example.out}
   \end{minipage}
   \end{tabular}}}
%%----------------------------------------------------------------------

%%----------------------------------------------------------------------
%% Idéntico al anterior pero sin frame
%%----------------------------------------------------------------------
\newenvironment{latexPlusCodeLeftnoFrame}
  {\VerbatimEnvironment
   \begin{VerbatimOut}{example.out}}
  {\end{VerbatimOut}
   \setlength{\parindent}{0pt}
   \begin{tabular}{ll}
   \begin{minipage}{0.55\linewidth}
     \inputminted[fontsize=\scriptsize,resetmargins]{latex}{example.out}
   \end{minipage} &
   \begin{minipage}{0.35\linewidth}
     \setlength{\parskip}{6pt plus 1pt minus 1pt}%
     \raggedright\scriptsize\input{example.out}
   \end{minipage}
   \end{tabular}}
%%----------------------------------------------------------------------

%%----------------------------------------------------------------------
%% Ambiente para generar el resultado de un código LaTeX seguido por 
%% el código
%%----------------------------------------------------------------------
\newenvironment{latexPlusCodeBelow}
  {\VerbatimEnvironment
   \begin{VerbatimOut}{example.out}}
  {\end{VerbatimOut}
		\input{example.out}
    \inputminted[fontsize=\scriptsize,resetmargins]{latex}{example.out}
   }
%%----------------------------------------------------------------------

%%------------------------------------------
%% Colores
%%------------------------------------------
\definecolorset{HTML}{}{}{ %
	color1,423E3A;color2,ACB8F1;color3,FF1E12;color4,FF7600;color5,FF8012;myBlue,027FDF; %
	myBlack,181818;myRed,AA3939;myGreen,19B170;myGray,181818;myOrange,FF6302;negative,181818; %
	positive,AA3939;enlace,1D8AEC;url,000000;youtube,C4302B;codeback,A4DCF2;codeframe,0B2C39; %
	structColor,201090;barColor,202122;backColor,8998B5;titlesColor,008100;blueR,1042A6; %
	blueOctave,015794;hlOrange,FFAD7E;citeColor,AA3939 %
}

\definecolorset{rgb}{}{}{ %
	signBlue,0, .25, .53;emeraldGreen,0, .79, .34;topaz,0, .6, .88; %
	indigoDye,.05, .31, .55;indigo2,.13, .53, .41;blueSpider,.15, .27, .43; %
	darkOliveGreen2,.74, .93, .41;ochre,.8, .47, .13;darkOrange,.8, .4, .0; %
	skyblue6,.2, .6, .8;firebrick,.7, .13, .13;blueice,.85, .96, .94; %
	lightcopper,.93, .76, .58
}
%%------------------------------------------}.
		
	\section{Requerimientos}
		El archivo \bftt{LaTeX2Code.tex} carga todos los paquetes necesarios para el uso de estos macros y define unos márgenes amplios para el documento que preferiblemente debe tratarse de un documento de la clase \bftt{article}. Con un preámbulo tal como el presentado en seguida estará listo para documentar cualquier ejemplo de \LaTeX{}.
		\begin{minted}{latex}
\documentclass{article}

\RequirePackage[utf8]{inputenc}
\RequirePackage[spanish]{babel}
\RequirePackage{amsmath, amsfonts, amssymb}
\RequirePackage{minted}
\RequirePackage{subcaption} 
\RequirePackage{wrapfig}
\RequirePackage{mdframed}
\RequirePackage{listings}
\RequirePackage{etex}
\RequirePackage{fancyhdr}
\RequirePackage{graphics}
\RequirePackage{xcolor}
\RequirePackage{tikz}
\usetikzlibrary{shadows}
\lfoot{Este ejemplo esta puesto a disposición mediante la licencia GPLv3, usted es libre de usarlo, estudiarlo, modificarlo o compartirlo siempre que las obras derivadas las comparta bajo los mismos términos y mencione al autor original.}
\pagenumbering{gobble}
\renewcommand{\headrulewidth}{0pt}

\graphicspath{{./images/}}

%%------------------------------------------
%% Márgenes
%%------------------------------------------
\RequirePackage[top = 1.5cm, bottom = 1.5cm, left = 1.5cm, right = 1.5cm, foot = .5cm]{geometry}
%%------------------------------------------

%%----------------------------------------------------------------------
%% Configuración global del ambiente minted para LaTeX
%%----------------------------------------------------------------------
\setminted[latex]{
style = friendly, 
frame = lines, 
bgcolor = codeback!10, 
linenos = true,
numberblanklines = false,
tabsize = 2, 
fontsize = \small,
resetmargins, 
framesep = 2mm, 
baselinestretch = 1.2,
numbersep = 3pt
}
%%----------------------------------------------------------------------

%%----------------------------------------------------------------------
%% Entradas:
%% #1 un color
%% #2 Un texto
%%----------------------------------------------------------------------
%% Salida: El texto #2 escrito en negrita y color #1
%%----------------------------------------------------------------------
\newcommand{\bfcolor}[2]{
\textbf{\textcolor{#1}{#2}}
}
%%----------------------------------------------------------------------
  
%%----------------------------------------------------------------------
%% Entradas:
%% #1 Un comando
%% Salida:
%%----------------------------------------------------------------------
%% El comando #1 en formato verbatim y negrilla
%%----------------------------------------------------------------------
\newcommand{\bftt}[1]{\textbf{\texttt{#1}}}
%%----------------------------------------------------------------------

%%----------------------------------------------------------------------
%% Entrada:
%% #1 Un caracter del teclado
%%----------------------------------------------------------------------
%% Salida:
%% La representación de la tecla
%%----------------------------------------------------------------------
\newcommand*\keystroke[1]{
  \tikz[baseline=(key.base)]
    \node[%
      draw,
      fill=white,
      drop shadow={shadow xshift=0.25ex,shadow yshift=-0.25ex,fill=black,opacity=0.75},
      rectangle,
      rounded corners=2pt,
      inner sep=1pt,
      line width=0.5pt,
      font=\scriptsize\sffamily
    ](key) {#1\strut}
  ;
}
%%----------------------------------------------------------------------

%%----------------------------------------------------------------------
%% Entradas:
%% #1 un caracter del teclado
%% Salida:
%% La representación de la tecla en negrilla
%%----------------------------------------------------------------------
\newcommand{\keystrokebftt}[1]{\keystroke{\bftt{#1}}}
%%----------------------------------------------------------------------

%%----------------------------------------------------------------------
%% Escribir un comando (LaTeX) no precedido por backslash
%%----------------------------------------------------------------------
\newcommand{\cmd}[1]{{\color[HTML]{008000}\bftt{#1}}} 
%%----------------------------------------------------------------------

%%----------------------------------------------------------------------
%% Escribe el caracter backslash
%%----------------------------------------------------------------------
\newcommand{\bs}{\char`\\}
%%----------------------------------------------------------------------

%%----------------------------------------------------------------------
%% Escribe un comando (LaTeX) precedido de backslash
%%----------------------------------------------------------------------
\newcommand{\cmdbs}[1]{\cmd{\bs{}#1}} 
%%----------------------------------------------------------------------

%%----------------------------------------------------------------------
%% Comandos requeridos por \cmdbegin y \cmdend
%%----------------------------------------------------------------------
\newcommand{\lcb}{\char '173}
\newcommand{\rcb}{\char '175}
%%----------------------------------------------------------------------

%%----------------------------------------------------------------------
%% Escriben el inicio y fin - respectivamente - de un ambiente LaTeX
%%----------------------------------------------------------------------
\newcommand{\cmdbegin}[1]{\cmdbs{begin\lcb}\bftt{#1}\cmd{\rcb}}
\newcommand{\cmdend}[1]{\cmdbs{end\lcb}\bftt{#1}\cmd{\rcb}}
%%----------------------------------------------------------------------

%%----------------------------------------------------------------------
%% Ambiente para generar código LaTeX y su resultado uno al lado del 
%% otro enmarcados en un frame sensillo
%%----------------------------------------------------------------------
\newenvironment{latexPlusCodeLeftFrame}
  {\VerbatimEnvironment
   \begin{VerbatimOut}{example.out}}
  {\end{VerbatimOut}
   \setlength{\parindent}{0pt}
   \fbox{\begin{tabular}{l|l}
   \begin{minipage}{0.55\linewidth}
     \inputminted[fontsize=\scriptsize,resetmargins]{latex}{example.out}
   \end{minipage} &
   \begin{minipage}{0.35\linewidth}
     \setlength{\parskip}{6pt plus 1pt minus 1pt}%
     \raggedright\scriptsize\input{example.out}
   \end{minipage}
   \end{tabular}}}
%%----------------------------------------------------------------------

%%----------------------------------------------------------------------
%% Idéntico al anterior pero sin frame
%%----------------------------------------------------------------------
\newenvironment{latexPlusCodeLeftnoFrame}
  {\VerbatimEnvironment
   \begin{VerbatimOut}{example.out}}
  {\end{VerbatimOut}
   \setlength{\parindent}{0pt}
   \begin{tabular}{ll}
   \begin{minipage}{0.55\linewidth}
     \inputminted[fontsize=\scriptsize,resetmargins]{latex}{example.out}
   \end{minipage} &
   \begin{minipage}{0.35\linewidth}
     \setlength{\parskip}{6pt plus 1pt minus 1pt}%
     \raggedright\scriptsize\input{example.out}
   \end{minipage}
   \end{tabular}}
%%----------------------------------------------------------------------

%%----------------------------------------------------------------------
%% Ambiente para generar el resultado de un código LaTeX seguido por 
%% el código
%%----------------------------------------------------------------------
\newenvironment{latexPlusCodeBelow}
  {\VerbatimEnvironment
   \begin{VerbatimOut}{example.out}}
  {\end{VerbatimOut}
		\input{example.out}
    \inputminted[fontsize=\scriptsize,resetmargins]{latex}{example.out}
   }
%%----------------------------------------------------------------------

%%------------------------------------------
%% Colores
%%------------------------------------------
\definecolorset{HTML}{}{}{ %
	color1,423E3A;color2,ACB8F1;color3,FF1E12;color4,FF7600;color5,FF8012;myBlue,027FDF; %
	myBlack,181818;myRed,AA3939;myGreen,19B170;myGray,181818;myOrange,FF6302;negative,181818; %
	positive,AA3939;enlace,1D8AEC;url,000000;youtube,C4302B;codeback,A4DCF2;codeframe,0B2C39; %
	structColor,201090;barColor,202122;backColor,8998B5;titlesColor,008100;blueR,1042A6; %
	blueOctave,015794;hlOrange,FFAD7E;citeColor,AA3939 %
}

\definecolorset{rgb}{}{}{ %
	signBlue,0, .25, .53;emeraldGreen,0, .79, .34;topaz,0, .6, .88; %
	indigoDye,.05, .31, .55;indigo2,.13, .53, .41;blueSpider,.15, .27, .43; %
	darkOliveGreen2,.74, .93, .41;ochre,.8, .47, .13;darkOrange,.8, .4, .0; %
	skyblue6,.2, .6, .8;firebrick,.7, .13, .13;blueice,.85, .96, .94; %
	lightcopper,.93, .76, .58
}
%%------------------------------------------

\author{Su Nombre}
\title{Un título atractivo}
\date{\today}
		\end{minted}
		
	\section{Comandos}
	
	\subsection{\cmdbs{bfcolor}}
		Este comando recibe como entradas un color y un texto y genera el texto escrito en negrilla y del color indicado. e.g \mintinline{latex}{\bfcolor{cyan!50!gray}{Este es un texto}} genera \bfcolor{cyan!50!gray}{Este es un texto}.
		
	\subsection{\cmdbs{bftt}}
		Toma como entrada un texto y genera el texto escrito en verbatim y negrilla. e.g. \mintinline{latex}{\bftt{Texto en verbatim y negrilla}} genera \bftt{Texto en verbatim y negrilla}. Este comando es ideal para resaltar nombres de archivos o partes de código que no necesariamente son comandos o ambientes.

	\subsection{\cmdbs{keystroke}}
		Recibe un carácter del teclado y genera una representación de la tecla correspondiente. e.g \mintinline{latex}{\keystroke{\%}} genera la representación \keystroke{\%}.
		
	\subsection{\cmdbs{keystrokebftt}}
		Identico a \cmdbs{keystroke} pero en negrilla. e.g \mintinline{latex}{\keystrokebftt{k}} genera \keystrokebftt{k}.
		
	\subsection{\cmdbs{cmd}}
		Ideal para escribir los nombres de comandos sin que estén precedidos por backslash. e.g. \mintinline{latex}{\cmd{operatorname}} genera \cmd{operatorname}.
		
	\subsection{\cmdbs{cmdbs}}
		Escribe el nombre de un comando precedido de backslash. e.g con \mintinline{latex}{\cmdbs{frac}} obtiene \cmdbs{frac}.
		
	\subsection{\cmdbs{bs}}	
		Escribe el carácter backslash.
		
	\subsection{\cmdbs{cmdbegin} y \cmdbs{cmdend}}	
		Escriben respectivamente el inicio y fin de cualquier ambiente \LaTeX{}. 
		
	\section{Ambientes}
	
	\subsection{\cmdbegin{latexPlusCodeLeftFrame}}
		Este ambienge genera una muestra de código \LaTeX{} seguida - a la derecha - por el resultado de su ejecución, todo enmarcado en un frame sencillo.
		
		\begin{latexPlusCodeLeftFrame}
\[
  \left|
    \begin{array}{ccc}
      1 & 2 & 3 \\
      4 & 5 & 6 \\
      7 & 8 & 0
    \end{array}
  \right|
\]
		\end{latexPlusCodeLeftFrame}
		
	\subsection{\cmdbegin{latexPlusCodeLeftnoFrame} y \cmdbegin{latexPlusCodeBelow}}
		El primero de estos actúa idéntico al anterior pero no dibuja el frame que enmarca al código y el resultado. El segundo elimina el frame y ubica el código debajo del resultado.
		
	\section{Otros}
		\subsection{\cmdbs{mintinline} y \cmdbegin{minted}}
			Este comando y \mintinline{latex}{\mintinline{latex}{code}} permiten agregar código \LaTeX{} en línea con el texto que no estén contemplados dentro de los macros de arriba y el ambiente \mintinline{latex}{\begin{minted}{latex}} hace lo propio pero con bloques de código.
		
		\subsection{Licencia}
			Agregando \mintinline{latex}{\thispagestyle{fancy}} justo después de \cmdbs{maketitle} se agrega una nota en el pie de la primera página indicando que el documento y su código fuente están disponible bajo los términos de la licencia LPPL v1.3c.
		
	\section{Licencia}
		\bftt{LaTeX2Code} es un conjunto de macros publicado mediante la licencia LPPL v1.3c, usted es libre de usarlo, estudiarlo, modificarlo o compartirlo siempre que las obras derivadas las comparta bajo los mismos términos y mencione al autor original.
\end{document}
