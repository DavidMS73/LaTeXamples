\documentclass{article}

\RequirePackage[utf8]{inputenc}
\RequirePackage[spanish]{babel}
\RequirePackage{amsmath, amsfonts, amssymb}
\RequirePackage{minted}
\RequirePackage{subcaption} 
\RequirePackage{wrapfig}
\RequirePackage{mdframed}
\RequirePackage{listings}
\RequirePackage{etex}
\RequirePackage{fancyhdr}
\RequirePackage{graphics}
\RequirePackage{xcolor}
\lfoot{Este ejemplo esta puesto a disposición mediante la licencia GPLv3, usted es libre de usarlo, estudiarlo, modificarlo o compartirlo siempre que las obras derivadas las comparta bajo los mismos términos y mencione al autor original.}
\pagenumbering{gobble}
\renewcommand{\headrulewidth}{0pt}

\graphicspath{{./images/}}

%%------------------------------------------
%% Márgenes
%%------------------------------------------
\RequirePackage[top = 1.5cm, bottom = 1.5cm, left = 1.5cm, right = 1.5cm, foot = .5cm]{geometry}
%%------------------------------------------

%%----------------------------------------------------------------------
%% Configuración global del ambiente minted para LaTeX
%%----------------------------------------------------------------------
\setminted[latex]{
style = friendly, 
frame = lines, 
bgcolor = codeback!10, 
linenos = true,
numberblanklines = false,
tabsize = 2, 
fontsize = \small,
resetmargins, 
framesep = 2mm, 
baselinestretch = 1.2,
numbersep = 3pt
}
%%----------------------------------------------------------------------

%%----------------------------------------------------------------------
%% Entradas:
%% #1 un color
%% #2 Un texto
%%----------------------------------------------------------------------
%% Salida: El texto #2 escrito en negrita y color #1
%%----------------------------------------------------------------------
\newcommand{\bfcolor}[2]{
\textbf{\textcolor{#1}{#2}}
}
%%----------------------------------------------------------------------
  
%%----------------------------------------------------------------------
%% Entradas:
%% #1 Un comando
%% Salida:
%%----------------------------------------------------------------------
%% El comando #1 en formato verbatim y negrilla
%%----------------------------------------------------------------------
\newcommand{\bftt}[1]{\textbf{\texttt{#1}}}
%%----------------------------------------------------------------------

%%----------------------------------------------------------------------
%% Entrada:
%% #1 Un caracter del teclado
%%----------------------------------------------------------------------
%% Salida:
%% La representación de la tecla
%%----------------------------------------------------------------------
\newcommand*\keystroke[1]{
  \tikz[baseline=(key.base)]
    \node[%
      draw,
      fill=white,
      drop shadow={shadow xshift=0.25ex,shadow yshift=-0.25ex,fill=black,opacity=0.75},
      rectangle,
      rounded corners=2pt,
      inner sep=1pt,
      line width=0.5pt,
      font=\scriptsize\sffamily
    ](key) {#1\strut}
  ;
}
%%----------------------------------------------------------------------

%%----------------------------------------------------------------------
%% Entradas:
%% #1 un caracter del teclado
%% Salida:
%% La representación de la tecla en negrilla
%%----------------------------------------------------------------------
\newcommand{\keystrokebftt}[1]{\keystroke{\bftt{#1}}}
%%----------------------------------------------------------------------

%%----------------------------------------------------------------------
%% Escribir un comando (LaTeX) no precedido por backslash
%%----------------------------------------------------------------------
\newcommand{\cmd}[1]{{\color[HTML]{008000}\bftt{#1}}} 
%%----------------------------------------------------------------------

%%----------------------------------------------------------------------
%% Escribe el caracter backslash
%%----------------------------------------------------------------------
\newcommand{\bs}{\char`\\}
%%----------------------------------------------------------------------

%%----------------------------------------------------------------------
%% Escribe un comando (LaTeX) precedido de backslash
%%----------------------------------------------------------------------
\newcommand{\cmdbs}[1]{\cmd{\bs{}#1}} 
%%----------------------------------------------------------------------

%%----------------------------------------------------------------------
%% Comandos requeridos por \cmdbegin y \cmdend
%%----------------------------------------------------------------------
\newcommand{\lcb}{\char '173}
\newcommand{\rcb}{\char '175}
%%----------------------------------------------------------------------

%%----------------------------------------------------------------------
%% Escriben el inicio y fin - respectivamente - de un ambiente LaTeX
\newcommand{\cmdbegin}[1]{\cmdbs{begin\lcb}\bftt{#1}\cmd{\rcb}}
\newcommand{\cmdend}[1]{\cmdbs{end\lcb}\bftt{#1}\cmd{\rcb}}
%%----------------------------------------------------------------------

%%----------------------------------------------------------------------
%% Ambiente para generar código LaTeX y su resultado uno al lado del 
%% otro enmarcados en un frame sensillo
%%----------------------------------------------------------------------
\newenvironment{exampletwouptiny}
  {\VerbatimEnvironment
   \begin{VerbatimOut}{example.out}}
  {\end{VerbatimOut}
   \setlength{\parindent}{0pt}
   \fbox{\begin{tabular}{l|l}
   \begin{minipage}{0.55\linewidth}
     \inputminted[fontsize=\scriptsize,resetmargins]{latex}{example.out}
   \end{minipage} &
   \begin{minipage}{0.35\linewidth}
     \setlength{\parskip}{6pt plus 1pt minus 1pt}%
     \raggedright\scriptsize\input{example.out}
   \end{minipage}
   \end{tabular}}}
%%----------------------------------------------------------------------

%%----------------------------------------------------------------------
%% Idéntico al anterior pero sin frame
%%----------------------------------------------------------------------
\newenvironment{exampletwouptinynoframe}
  {\VerbatimEnvironment
   \begin{VerbatimOut}{example.out}}
  {\end{VerbatimOut}
   \setlength{\parindent}{0pt}
   \begin{tabular}{ll}
   \begin{minipage}{0.55\linewidth}
     \inputminted[fontsize=\scriptsize,resetmargins]{latex}{example.out}
   \end{minipage} &
   \begin{minipage}{0.35\linewidth}
     \setlength{\parskip}{6pt plus 1pt minus 1pt}%
     \raggedright\scriptsize\input{example.out}
   \end{minipage}
   \end{tabular}}
%%----------------------------------------------------------------------

%%----------------------------------------------------------------------
%% Ambiente para generar el resultado de un código LaTeX seguido por 
%% el código
%%----------------------------------------------------------------------
\newenvironment{latexPlusCode}
  {\VerbatimEnvironment
   \begin{VerbatimOut}{example.out}}
  {\end{VerbatimOut}
		\input{example.out}
    \inputminted[fontsize=\scriptsize,resetmargins]{latex}{example.out}
   }
%%----------------------------------------------------------------------

%%------------------------------------------
%% Colores
%%------------------------------------------
\definecolorset{HTML}{}{}{ %
	color1,423E3A;color2,ACB8F1;color3,FF1E12;color4,FF7600;color5,FF8012;myBlue,027FDF; %
	myBlack,181818;myRed,AA3939;myGreen,19B170;myGray,181818;myOrange,FF6302;negative,181818; %
	positive,AA3939;enlace,1D8AEC;url,000000;youtube,C4302B;codeback,A4DCF2;codeframe,0B2C39; %
	structColor,201090;barColor,202122;backColor,8998B5;titlesColor,008100;blueR,1042A6; %
	blueOctave,015794;hlOrange,FFAD7E;citeColor,AA3939 %
}

\definecolorset{rgb}{}{}{ %
	signBlue,0, .25, .53;emeraldGreen,0, .79, .34;topaz,0, .6, .88; %
	indigoDye,.05, .31, .55;indigo2,.13, .53, .41;blueSpider,.15, .27, .43; %
	darkOliveGreen2,.74, .93, .41;ochre,.8, .47, .13;darkOrange,.8, .4, .0; %
	skyblue6,.2, .6, .8;firebrick,.7, .13, .13;blueice,.85, .96, .94; %
	lightcopper,.93, .76, .58
}
%%------------------------------------------
\input{Tikz2Code.tex}

\title{Manual \bftt{TikZ2Code.tex}}
\author{MSc. Fausto M. Lagos S.}
\date{\today}

\begin{document}
\maketitle

\begin{abstract}
	\bftt{TikZ2Code.tex} es un conjunto de macros desarrollados para documentar la construcción de figuras con el paquete \bftt{TikZ}, estos macros permiten desarrollar el código de la figura \bftt{TikZ} y agregar de forma automática ésta en un ambiente \bftt{figure} seguido del código o generar un archivo auxilar con el código para insertarlo en el documento en la posición elegida por el autor.
\end{abstract}

\section{A modo de instalación}

Para utilizar los macros \bftt{TikZ2code} no hace falta realizar ningún procedimiento de instalación, basta con incluir \mintinline{latex}{%%----------------------------------------------------------------------
%% TikZ2Code.tex
%% Copyright 2020 Fausto M. Lagos S. - @piratax007
%% piratax007@protonmail.ch
%%
%% Este trabajo puede ser usado, estudiado, modificado y 
%% redistribuido bajo los términos de la Licencia LaTeX Project Public License v1.3c+. 
%% La última versión de esta licencia puede consultarse en
%% http://www.latex-project.org/lppl.txt
%%----------------------------------------------------------------------

\RequirePackage{tikz}
\usetikzlibrary{babel, calc, positioning, fit, arrows, external, shadows}
\RequirePackage{graphicx}
\RequirePackage{tcolorbox}
\tcbuselibrary{listings, minted, breakable, skins}
\RequirePackage{tkz-euclide}
\RequirePackage{tkz-fct}
% puede agregar los paquete tikz que requiera

%%----------------------------------------------------------------------
%% Configuración global del ambiente minted para LaTeX
%%----------------------------------------------------------------------
\setminted[latex]{
style = friendly, 
frame = lines, 
bgcolor = codeback!10, 
linenos = true,
numberblanklines = false,
tabsize = 2, 
fontsize = \small,
resetmargins, 
framesep = 2mm, 
baselinestretch = 1.2,
numbersep = 3pt
}
%%----------------------------------------------------------------------

%%----------------------------------------------------------------------
%% Entradas
%% El subtítulo para la figura
%%----------------------------------------------------------------------
%% Salidas - La figura seguida del código LaTeX que la desarrolla
%%----------------------------------------------------------------------
\newenvironment{tikzPlusCode}[1][]
  {\def\tempCaption{#1}
  \VerbatimEnvironment
   \begin{VerbatimOut}{figure.out}}
  {\end{VerbatimOut}
  \begin{figure}[H]
   	 \centering
     \input{figure.out}
     \caption{\tempCaption}
   \end{figure}  
   \setlength{\parindent}{0pt}
   \begin{minipage}{\textwidth}
     \inputminted{latex}{figure.out}
   \end{minipage}
  }
%%----------------------------------------------------------------------

%%----------------------------------------------------------------------
%% Entradas:
%% Obligatorias - Una cadena ID del código - el nombre del archivo out 
%% que contiene el
%% código LaTeX correspondiente a la figura
%%----------------------------------------------------------------------
%% Salidas: La figura y un archivos ID.out que contiene el código 
%% LaTeX de la figura
%%----------------------------------------------------------------------
\newenvironment{tikzToCode}[1][]
  {\def\tempId{#1}
  \VerbatimEnvironment
    \begin{VerbatimOut}{\tempId.out}}
  {\end{VerbatimOut}
     \input{\tempId.out}
  }
%%----------------------------------------------------------------------

%\tikzexternalize[prefix = figures/] 
%\usetkzobj{all}
% %%----------------------------------------------------------------------
% hace \tkzDrawLine compatible con la librería babel de Tikz
%\makeatletter
%\patchcmd{\tkz@DrawLine}{\begingroup}{\begingroup\makeatletter}{}{}
%\makeatother
% %%----------------------------------------------------------------------

%%------------------------------------------
%% Colores
%%------------------------------------------
\definecolorset{HTML}{}{}{ %
	color1,423E3A;color2,ACB8F1;color3,FF1E12;color4,FF7600;color5,FF8012;myBlue,027FDF; %
	myBlack,181818;myRed,AA3939;myGreen,19B170;myGray,181818;myOrange,FF6302;negative,181818; %
	positive,AA3939;enlace,1D8AEC;url,000000;youtube,C4302B;codeback,A4DCF2;codeframe,0B2C39; %
	structColor,201090;barColor,202122;backColor,8998B5;titlesColor,008100;blueR,1042A6; %
	blueOctave,015794;hlOrange,FFAD7E;citeColor,AA3939 %
}

\definecolorset{rgb}{}{}{ %
	signBlue,0, .25, .53;emeraldGreen,0, .79, .34;topaz,0, .6, .88; %
	indigoDye,.05, .31, .55;indigo2,.13, .53, .41;blueSpider,.15, .27, .43; %
	darkOliveGreen2,.74, .93, .41;ochre,.8, .47, .13;darkOrange,.8, .4, .0; %
	skyblue6,.2, .6, .8;firebrick,.7, .13, .13;blueice,.85, .96, .94; %
	lightcopper,.93, .76, .58
}
%%------------------------------------------} - teniendo el correspondiente archivo en la raíz del proyecto - en el preámbulo para que se disponga de los ambientes \cmdbegin{tikzPlusCode} y \cmdbegin{tikzToCode}

\section{Uso del ambiente \bftt{tikzPlusCode}}

Este ambiente crea una figura seguida del código \textbf{TikZ} requerido para obtenerla.

\begin{exampletwouptinynoframe}
\begin{tikzPlusCode}[Uso de \cmdbs{foreach}]
\begin{tikzpicture}
  \foreach \x in {1, 2, ..., 5}{
    \draw (\x, 0) circle[radius = 1cm];
  }
\end{tikzpicture}
\end{tikzPlusCode}
\end{exampletwouptinynoframe}

El ambiente \bftt{tikzPlusCode} recibe como parámetro de entrada obligatorio el subtítulo (caption) de la figura, en su interior se define un ambiente \bftt{tikzpicture} con el código de la figura.

\section{Uso del ambiente \bftt{tikzToCode}}

El ambiente \bftt{tikzToCode} recibe como parámetro de entrada obligatorio un ID para identificar el archivo \bftt{.out} que contendrá el código de la figura \bftt{TikZ}. Este ambiente puede ubicarse o no dentro de un ambiente figure y para presentar el código correpondiente a la figura \bftt{TikZ} debe usarse el comando \mintinline{latex}{\inputminted{latex}{ID.out}} en cualquier parte del documento.

\subsection{Obtensión de la gráfica \bftt{TikZ} con el ambiente \bftt{tikzToCode}}

\begin{exampletwouptinynoframe}
\begin{tikzToCode}[idCode]
\begin{tikzpicture}
  \foreach \x in {1, 2, ..., 5}{
    \draw (\x, 0) circle[radius = 1cm];
  }
\end{tikzpicture}	
\end{tikzToCode}
\end{exampletwouptinynoframe}

\subsection{Obtensión del código generado con el ambiente \bftt{tikzToCode}}

\begin{exampletwouptinynoframe}
\inputminted{latex}{idCode.out}
\end{exampletwouptinynoframe}

\begin{latexPlusCode}
esta es una prueba $x^2$
\end{latexPlusCode}
\end{document}